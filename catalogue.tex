\documentclass[a4paper,10pt]{article}
\usepackage[french]{babel} % réglage de la langue
\usepackage[utf8]{inputenc} % permet de taper des caractères accentués dans le fichier source
\usepackage[T1]{fontenc} % permet d'afficher correctement les caractères accentués, avec césures etc
\usepackage{lmodern} % rétablissement des polices vectorielles
\usepackage[margin=1cm]{geometry}
\usepackage{amssymb,amsthm,amsmath,mathrsfs,stmaryrd,multicol,comment,url} 
\usepackage{tikz}
%\usepackage[francais,bloc,ordre]{automultiplechoice} % option 'bloc' pour ne pas changer de page dans une question, et 'ordre' pour ne pas permuter les réponses.


\newcommand\pad[1]{\ifnum #1 < 1000 0\fi \ifnum #1 < 100 0\fi \ifnum #1 < 10 0\fi #1} % un peu moche mais bon

\newenvironment{reponses}{}{}
\newcommand{\bonne}[1]{\dotfill #1}
\newcommand{\mauvaise}[1]{}



% packages et macros pour questions Maxime :
\usepackage{mathdots}
\newcommand{\R}{\mathbb R}
\newcommand{\C}{\mathbb C}
\newcommand{\N}{\mathbb N}
\newcommand{\Z}{\mathbb Z}
\newcommand{\eq}[1]{\underset{#1}{\sim}}
\newcommand{\cvg}[1]{\xrightarrow[#1]{}}
\DeclareMathOperator{\tr}{Tr}
\DeclareMathOperator{\id}{Id}
\DeclareMathOperator{\rg}{rg}
\DeclareMathOperator{\im}{Im}
\DeclareMathOperator{\Mat}{Mat}




\begin{document}
\title{Catalogue}
\author{Damien Mégy}
\maketitle
Cet document affiche le catalogue de tous les vrai-faux disponibles à l'adresse \url{https://github.com/exo7math/quiz-exo7} de manière compacte, pour une prévisualisation rapide. Pour plus d'exemples d'utilisation, voir le sous-dossier \og exemples/\fg.

Ces questions sont celles de l'application de vrai-faux \url{https://dmegy.perso.math.cnrs.fr/quiz}.

Les questions sont en vrac. Des listes thématiques seront disponibles dans le sous-dossier \og listes\fg{} du repo github. 

\vspace{1cm}

\foreach \n in {1, ..., 1600} {%
	\IfFileExists{latex_amc/10\pad{\n}.tex}{%
		\noindent \textbf{10\pad{\n}.tex --- }
		\input{latex_amc/10\pad{\n}.tex}%
	}{}%
}%
\end{document}