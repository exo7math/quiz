<b>Énoncé</b> : déterminer le domaine de définition de $\sqrt{-1+x-x^2}$.<br> <b>Solution rédigée à évaluer :</b><br>  «Soit $x\in\mathbb{R}$.  L'expression $\sqrt{-1+x-x^2}$ est bien définie si et seulement si $-1+x-x^2\geq 0$. Ce trinôme a un discriminant égal à $\Delta=b^2-4ac=-3$ donc n'a aucune racine réelle. Il ne s'annule donc jamais et donc est toujours positif. Le domaine de définition de $\sqrt{-1+x-x^2}$ est donc $\mathbb{R}$ tout entier.»

\begin{reponses}
\mauvaise{Vrai}
\bonne{Faux}
\end{reponses}

\begin{comment}
Discriminant correct mais le trinôme est négatif.
\end{comment}

